\documentclass[12pt,a4paper]{article}
\usepackage[x11names]{xcolor}
\usepackage[T1]{fontenc}
\usepackage{xltxtra}
\usepackage{xunicode}

%\usepackage[top=1.2in,bottom=1.2in,left=1.2in,right=1in]{geometry} % 页边距
\defaultfontfeatures{Mapping=tex-text}%连字号

\usepackage[boldfont,slantfont,CJKchecksingle]{xeCJK} % 允许斜体和粗体
\xeCJKsetup{PunctStyle=hangmobanjiao}
\linespread{1.2}                       % 行间距
\setlength{\parskip}{\baselineskip}     % 段间距

\XeTeXlinebreaklocale "zh"
\XeTeXlinebreakskip = 0pt plus 1pt minus 0.1pt

\usepackage[unicode=true,colorlinks,linkcolor=blue]{hyperref} % 超链接
\setCJKmainfont[BoldFont=XinGothic-ZhangYue W4, ItalicFont=方正粗楷简体]{方正新书宋_GBK}
\setCJKmonofont[BoldFont=XinGothic-ZhangYue W6]{方正粗楷简体}
\setmainfont{Noto Serif}                    % 英文衬线字体
\setmonofont{DejaVu Sans Mono}              % 英文等宽字体
\setsansfont{Noto Sans}                     % 英文无衬线字体


\usepackage{graphicx}                   % 嵌入png图像
\usepackage{longtable,tabu,booktabs}
\usepackage{pdflscape}

\usepackage{tocloft}                        % 目录
\renewcommand\contentsname{目录}
\renewcommand{\today}{\number\year 年 \number\month 月 \number\day 日} % 中文日期

\usepackage{amsmath}
\numberwithin{equation}{section}

\usepackage{amsthm}

\theoremstyle{plain}
\newtheorem{thm}{定理}[section] % reset theorem numbering for each chapter

\theoremstyle{definition}
\newtheorem{defn}[thm]{定义} % definition numbers are dependent on theorem numbers
\newtheorem{exmp}[thm]{示例} % same for example numbers

\usepackage[linecolor=red,bordercolor=red,backgroundcolor=white,textwidth=5em,textsize=footnotesize]{todonotes}

\newcommand\abs[1]{\left|#1\right|}

\begin{document}

\title{\textbf{我是文档标题}}

\author{卿培 <hello@qingpei.me> \\
        站长~\\
        qingpei.me
}

\date{\today}


\maketitle
\clearpage

\tableofcontents
\clearpage
\listoftodos

\section{图像采集} % (fold)
\label{sec:capture}

待整理。

% section 图像采集 (end)

\section{显示质量评价} % (fold)
\label{sec:evaluation}

目前我们关注两个方面的显示质量:\emph{亮度均匀性}和\emph{色彩准确度}。

\subsection{亮度均匀性} % (fold)
\label{sub:luminance_uniformity}

\begin{defn}
    全彩屏上一个像素$P$的\emph{亮度}$L_P$由采集到的照片中该像素范围内各个像素$p_i$在CIELAB空间$L$通道的几何平均值\footnote{相比于算术平均,几何平均对少量噪点更不敏感。}表示。

    \begin{equation}
        L_P = \left(\prod_{p_i \in P} L_{p_i} \right)^{1/n}
    \end{equation}
\end{defn}

\begin{defn}
    全彩屏的亮度均匀性为其中最大亮度像素与最小亮度像素的亮度比。

    \begin{equation}
        Luminance\ uniformity = \frac{L_{max}}{L_{min}}
    \end{equation}
\end{defn}

理想的亮度均匀性为$1$。这个比值越大,表示亮度均匀性越差。

这时需要一个检验标准来判定亮度均匀性是否合格,这个标准可以是$1.1$,可以是$1.3$,当然也可以是$1.05$等很严格的数值。\todo{了解这方面的现有标准}

另外一种评价方式是基于区域的。将一个显示屏分为$N$个区域,每个区域求出各自的平均亮度$L_i$,所有区域有一个全局的平均亮度$L_{avg}$,此时给出一个评价标准$\delta$,要求各个区域都符合标准。

\begin{equation}
    \abs{L_i - L_{avg}} < \delta
\end{equation}

同样,这里的$\delta$如何取值与上文一样需要商榷。

% subsection 亮度均匀性 (end)

\subsection{色彩准确度} % (fold)
\label{sub:color_accuracy}

\begin{defn}
    全彩屏上一个像素$P_i$的颜色$(L_{P_i},a_{P_i},b_{P_i})$为采集到的照片中该像素范围内各个像素$p_j$在CIELAB空间$L,a,b$通道上的几何平均值。
\end{defn}

\begin{defn}
    全彩屏上一个像素$P_i$显示某个参考色时的\emph{色差}$\Delta E_{P_i,color}$为该像素的颜色与参考色的颜色由CIE DE2000\footnote{CIE DE2000很复杂,如果放宽要求可以采用CIE94甚至CIE76定义的色差。}定义的色差。

\end{defn}

理想情况下色差为$0$,数值越大表示偏色越严重。

\begin{defn}
    全彩屏的色彩准确度分为两项,一是平均色差$\Delta E_{avg,color}$,二是最大色差$\Delta E_{max,color}$。

    \begin{equation}
        \Delta E_{avg,color} = \frac{\sum_{i=1}^N{\Delta E_{P_i,color}}}{N}
    \end{equation}
    \begin{equation}
        \Delta E_{max,color} = max\left(\sum{\Delta E_{P,color}}\right)
    \end{equation}
\end{defn}

理想情况下色差为$0$。一般来说$\Delta E<2.3$属于被认为肉眼不可见。优化目标选择平均色差还是最大色差有待商榷。\todo{优化目标的选择也需要查一查显示器生产厂家是如何做的。}

% subsection 色彩准确度 (end)

% section 显示质量评价 (end)



\end{document}
