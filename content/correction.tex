\section{亮度校正} % (fold)
\label{sec:correction}

经过亮度均匀性评价,我们得到每个像素的亮度表示。由于HoughCircles搜索到的圆并非按顺序排列的,需要将其对应到行和列。

然后将亮度数据重排成矩阵形式 $B$。

对于亮度矩阵,我们可以得到一个全像素平均亮度 $L_{overall}$。此时可以求出一个校正矩阵 $C$ 使得

\begin{equation}
    C \times B = L_{overall} \cdot I
\end{equation}

将校正矩阵 $C$ 应用于传输给控制盒的输出信号,等价于在硬件端进行校正。假设我们发给屏的信号亮度为 $B_{orig}$,采集到屏实际输出是上文提到的 $B$,那么屏的输入输出转化矩阵即为 $K$,它满足 $B_{orig} \times K = B$。

由于矩阵乘法满足结合律,下式也同样成立:

\begin{equation}
    C \times B_{orig} \times K = C \times B = L_{overall} \cdot I
\end{equation}

这样就可以通过校正输入信号来预览校正结果了。

% section 亮度校正 (end)
