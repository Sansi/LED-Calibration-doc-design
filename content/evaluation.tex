\section{显示质量评价} % (fold)
\label{sec:evaluation}

目前我们关注两个方面的显示质量:\emph{亮度均匀性}和\emph{色彩准确度}。

\subsection{亮度均匀性} % (fold)
\label{sub:luminance_uniformity}

\begin{defn}
    全彩屏上一个像素$P$的\emph{亮度}$L_P$由采集到的照片中该像素范围内各个像素$p_i$在CIELAB空间$L$通道的几何平均值\footnote{相比于算术平均,几何平均对少量噪点更不敏感。}表示。

    \begin{equation}
        L_P = \left(\prod_{p_i \in P} L_{p_i} \right)^{1/n}
    \end{equation}
\end{defn}

\begin{defn}
    全彩屏的亮度均匀性为其中最大亮度像素与最小亮度像素的亮度比。

    \begin{equation}
        Luminance\ uniformity = \frac{L_{max}}{L_{min}}
    \end{equation}
\end{defn}

理想的亮度均匀性为$1$。这个比值越大,表示亮度均匀性越差。

这时需要一个检验标准来判定亮度均匀性是否合格,这个标准可以是$1.1$,可以是$1.3$,当然也可以是$1.05$等很严格的数值。\todo{了解这方面的现有标准}

另外一种评价方式是基于区域的。将一个显示屏分为$N$个区域,每个区域求出各自的平均亮度$L_i$,所有区域有一个全局的平均亮度$L_{avg}$,此时给出一个评价标准$\delta$,要求各个区域都符合标准。

\begin{equation}
    \abs{L_i - L_{avg}} < \delta
\end{equation}

同样,这里的$\delta$如何取值与上文一样需要商榷。

% subsection 亮度均匀性 (end)

\subsection{色彩准确度} % (fold)
\label{sub:color_accuracy}

\begin{defn}
    全彩屏上一个像素$P_i$的颜色$(L_{P_i},a_{P_i},b_{P_i})$为采集到的照片中该像素范围内各个像素$p_j$在CIELAB空间$L,a,b$通道上的几何平均值。
\end{defn}

\begin{defn}
    全彩屏上一个像素$P_i$显示某个参考色时的\emph{色差}$\Delta E_{P_i,color}$为该像素的颜色与参考色的颜色由CIE DE2000\footnote{CIE DE2000很复杂,如果放宽要求可以采用CIE94甚至CIE76定义的色差。}定义的色差。

\end{defn}

理想情况下色差为$0$,数值越大表示偏色越严重。

\begin{defn}
    全彩屏的色彩准确度分为两项,一是平均色差$\Delta E_{avg,color}$,二是最大色差$\Delta E_{max,color}$。

    \begin{equation}
        \Delta E_{avg,color} = \frac{\sum_{i=1}^N{\Delta E_{P_i,color}}}{N}
    \end{equation}
    \begin{equation}
        \Delta E_{max,color} = max\left(\sum{\Delta E_{P,color}}\right)
    \end{equation}
\end{defn}

理想情况下色差为$0$。一般来说$\Delta E<2.3$属于被认为肉眼不可见。优化目标选择平均色差还是最大色差有待商榷。\todo{优化目标的选择也需要查一查显示器生产厂家是如何做的。}

% subsection 色彩准确度 (end)

% section 显示质量评价 (end)
