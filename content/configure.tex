% !TEX root = ../The Design of Display Calibration Tool.tex

\section{找最佳拍摄参数} % (fold)
\label{sec:configure}

焦距需要手动在镜头上调整,目标是被拍摄的屏幕范围应尽可能充满取景框。这里所寻找的参数是光圈快门组合。ISO应使用相机允许的最低值,以尽量避免噪点的产生。

寻找最优拍摄的步骤如下:

\begin{enumerate}
	\item 根据经验,预置一个默认参数
	\item 屏幕显示纯色,根据校正目标亮度调整,例如校正目标是50\%亮度下均匀,则使用 (127,127,127)
	\item \label{item:cap} 采集一张图像,分析灰度直方图
	\item 若饱和像素超过 1\% 的总像素,降低亮度,回到 \ref{item:cap}
	\item 亮度中位数在 200 以下,提高亮度,回到 \ref{item:cap}
	\item 饱和像素比例低于 1\%,且亮度中位数大于 200,完成\footnote{这几个步骤里的具体数值需要实际拍摄后确定,这里的数字仅为举例}
\end{enumerate}

在界面上应允许如下操作

\begin{enumerate}
	\item 人工设定参数
	\item 跳过参数设置
\end{enumerate}

人工设定参数的选项从相机读取可用的光圈、快门之后给出。

% section 找最佳拍摄参数 (end)
